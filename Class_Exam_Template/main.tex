\documentclass[addpoints]{exam}
\usepackage[utf8]{inputenc}


\begin{document}
%This code creates the text before the first question
%-------------------------------------------------------------------
\begin{center}
\fbox{\fbox{\parbox{5.5in}{\centering
Answer the questions in the spaces provided. If you run out of room
for an answer, continue on the back of the page.}}}
\end{center}

\vspace{5mm}

\makebox[\textwidth]{Name and section:\enspace\hrulefill}

\vspace{5mm}

\makebox[\textwidth]{Instructor’s name:\enspace\hrulefill}
%-------------------------------------------------------------------

%Here, the questions begin
\begin{questions}

%First question below
\question Given the equation \(x^n + y^n = z^n\) for \(x,y,z\) and \(n\) positive
integers. 
%This question has several parts
\begin{parts}
\part[5] For what values of $n$ is the statement in the previous question true?
\vspace{\stretch{1}} %Equally distributes the available space

\part[2 \half] For $n=2$ there's a theorem with a special name. What's that name?
\vspace{\stretch{1}}


\bonuspart[2 \half] What famous mathematician had an elegant proof for this theorem but there was
not enough space in the margin to write it down?
\vspace{\stretch{1}}

\end{parts}

\droptotalpoints %Prints the number of points in this question

%The next two questions are multiple choice examples
%-------------------------------------------------------------------
\question Which of these famous physicists invented time?

\begin{oneparchoices}
 \choice Stephen Hawking 
 \choice Albert Einstein
 \choice Emmy Noether
 \choice This makes no sense
\end{oneparchoices}

\question Which of these famous physicists published a paper on Brownian Motion?

\begin{checkboxes}
 \choice Stephen Hawking 
 \choice Albert Einstein
 \choice Emmy Noether
 \choice I don't know
\end{checkboxes}
%-------------------------------------------------------------------

\question[20] Compute \[\int_{0}^{\infty} \frac{\sin(x)}{x}\]

\vspace{\stretch{1}}

\bonusquestion[30] Prove that the real part of all non-trivial zeros of the function 
\(\zeta(z)\) is \(\frac{1}{2}\) (A million-dollar question)
\vspace{\stretch{1}}

\end{questions}

\clearpage
\begin{center}
\combinedgradetable[h][questions]
\end{center}

\end{document}